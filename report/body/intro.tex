\section{项目概述}

这篇论文的核心是解决一个在线数据定价问题:卖家拥有大量同质数据点(都从一个独立同分布的分布中采样得到),希望通过设定价格曲线来出售数据以最大化长期收益;买家分为 $m$ 种类型,每种类型都有一个估值曲线 $v_i(n)$,表示购买 $n$ 个数据点获得的价值。

它讨论的问题与一般的的多臂老虎机以及完全信息反馈问题不同之处在于这是一个非对称反馈问题:

\begin{itemize}
    \item 在随机设置中,卖家事先不知道买方的类型分布。
    \item 在对抗性设置中,卖家事先不知道买方类型的序列。
\end{itemize}

买家知道自己的类型分布以及卖家设置的价格曲线。当买家购买后,会向卖家透露其类型信息,若买家决定不购买,则不会向卖家透露自己的任何类型信息。

卖家在设计价格曲线时需要进行一定的权衡:

\begin{itemize}
    \item \textbf{利用}:如果价格曲线较高,可能可以增加即时收益,但也可能会导致买家不购买,在无法获得任何收益的同时也无法获取买家的类型信息,这对于后续最大化长期利益不利。
    \item \textbf{探索}:如果价格曲线较低,可能会吸引更多买家购买,虽然能获取更多的类型信息,但可能会导致卖家的收益降低。
\end{itemize}

为了使这个问题可解,论文引入了以下假设(在实际数据和机器学习应用中通常成立):

\begin{itemize}
    \item \textbf{单调性,M}:买方估值曲线 $v_i(n)$ 是非递减的(即数据越多价值越大)
    \item \textbf{有限类型,F}:买方类型 $m$ 的数量有限
    \item \textbf{平滑性,S}:对于所有的 $n,\ n' \in [N]$,估值的变化满足 ${v_i(n+n') - v_i(n)} \leqslant \frac{L}{N}n'$。\\
        这表明买方的估值在购买少量额外数据点时不会剧烈改变。
    \item \textbf{收益递减,D}:存在某个 $J > 0$,使得对于所有类型 $i \in [m]$ 和所有 $n \in [N]$,有 $v_i(n+1) - v_i(n) \leqslant \frac{J}{n}$。\\
        这意味着额外数据点带来的边际价值会随着已拥有数据量的增加而递减,这与许多机器学习准确性曲线的特性一致。
\end{itemize}
论文的贡献在于提出了新的离散化方案和在线学习算法:

\begin{enumerate}
    \item 论文证明了在只有 $m$ 种类型的情况下,最优定价曲线是“简单”的,即任何非递减的价格曲线都可以用一个“$m$步”的阶梯跳跃价格曲线 $\bar{p}$ 来近似且其产生的预期收益至少相同,这极大地缩小了需要考虑的价格曲线空间。
    \item 论文还对不同的在线设置(随机设置与对抗性设置)设计了相应的算法(基于 UCB 和基于 FTPL),分别实现了 $\tilde{O}(m\sqrt{T})$ 和 $\tilde{O}(m^{3/2}\sqrt{T})$ 的(期望)遗憾界。
\end{enumerate}

虽然总体计算复杂度可能随买方类型数量 $m$ 增加而呈指数级增长,但论文提供的算法能保证在类型数量有限但数据集规模庞大时的计算可行性。
