\section{理论部分}

\subsection{基于 UCB 的动态定价算法}

在随机设置下,买方类型服从一个固定但未知的分布 $q$,其中面临的挑战包括:

\begin{enumerate}
    \item \text{价格曲线空间巨大}:价格曲线是一个函数,直接搜索所有可能的函数显然不可行,即使进行简单的离散化,价格曲线的数量也可能巨大,导致UCB算法的遗憾值(regret)对价格曲线空间的大小 $|\bar{\mathcal{P}}|$ 呈指数依赖,性能很差。
    \item \text{非对称反馈}:卖方只在买方购买数据时才能知道买方的类型,如果买家没有购买,卖方无法得知当前买方的类型信息。
\end{enumerate}

\subsubsection{设计思想}

论文对这两个挑战分别提出了解决的思路和方案:

\begin{enumerate}
    \item \text{离散化方案}:
        \begin{itemize}
            \item 论文首先证明了对于 $m$ 种买家类型,任何非递减的价格曲线都可以用一个“$m$步”的价格曲线 $\bar{p}$ 来近似,大大缩小了需要考虑的价格曲线空间。
            \item 论文还对估值空间(价格的范围)和数据数量空间进行了精细的分层离散化,尤其是在价格较低或数据量较少时更加密集,以捕捉重要的变化。
            \item 最后论文还进一步结合曲线的单调性、平滑性、收益递减性等,进一步缩小了离散化曲线集合 $\bar{\mathcal{P}}$ 的大小。
        \end{itemize}

        经过上述的离散化方案,虽然 $|\bar{\mathcal{P}}|$ 仍然可能很大,但它使得后续的UCB算法可以在这个有限集合上运行。

    \item \text{构建 UCB}:面对非对称反馈问题,论文的创新之处在于没有为每个价格曲线维护 UCB,而是为买家类型分布 $q$ 维护一个 UCB,然后再把 $q$ 的 UCB 转化为收益的 UCB。
    
        \begin{itemize}
            \item 在第 1 轮中,算法选择一个 0 价格曲线(任意数量数据的价格都为 0),以确保所有买家都会购买数据并透露其类型,从而获得初始的类型反馈信息。
            \item 记录在过去 $t$ 轮中类型 $i$ 的买方在理论上能够从已选择的价格 $p_{\tau}$ 中获得非负效用(即能进行购买)的次数 $T_{i, t}$
            \item 在类型 $i$ 会购买的那些轮次中,统计类型 $i$ 实际出现的经验频率 $\bar{q}_{i,t}$,然后利用 Hoeffding 不等式构建对真实频率 $q_i$ 的乐观估计上界 $\hat{q}_{i,t}$
            \item 最后利用卖方已知的估值曲线 $v_i(n)$ 和刚刚计算出的类型频率 UCB $\hat{q}_{i,t}$,计算出任意价格曲线 p 的预期收入的 UCB $\widehat{\text{rev}}(p) = \sum_{i=1}^{m} \hat{q}_{i,t} \cdot p(n_{i,P})$。算法在每轮中会选择离散化价格集合 $\bar{\mathcal{P}}$ 中预期收入 UCB 最大的价格曲线 $p_t$ 作为当前轮次的定价曲线。
        \end{itemize}
\end{enumerate}

\subsubsection{证明思路}

这个算法的遗憾可以被分解为离散误差和估计误差两部分

\begin{enumerate}
    \item 首先证明在上述的离散化方案中离散误差可以被近似到 $O(\varepsilon)$ 的范围内。
    \item 接着使用 Hoeffding 不等式证明 $|q_i - \hat{q}_{i,t}| \leqslant O(\sqrt{\frac{\log T}{T_{i,t}}})$,即 $\hat{q}_{i,t}$ 对 $q_i$ 估计误差以 $\sqrt{\frac{\log T}{T_{i,t}}}$ 的速度衰减。
    \item 最后可以证明估计误差不超过 $\tilde{O}(m\sqrt{T})$,计算两部分之和得到的遗憾界也不超过 $\tilde{O}(m\sqrt{T})$
\end{enumerate}

\subsection{基于 FTPL 的动态定价算法}

在对抗性设置中,买家类型的序列是由对手选择的,这相当于总是遇见最坏的情况,这要求设计一个对任意序列输入都有良好表现的算法。

\subsubsection{设计思想}

FTPL 通常用于完全信息问题,论文修改了 FTPL 中的“奖励” $r_t(p)$ 的定义来适应这个非对称反馈问题中的情况。

\begin{itemize}
    \item 发生购买时:卖家会得知买家的类型 $i_t$,此时所有价格曲线 $p$ 的奖励为在这一类型和当前价格 $p$ 下的真实收益 $r_t(p) = p(n_{i_t,p})$。
    \item 不发生购买时:卖家不知道本次买家的具体类型,但卖家知道哪些类型的买家一定会购买,这些类型的集合记为 $S_t$,因此本次买家的类型一定属于 $S_t^c$。\\
        算法此时将奖励定义为在价格 $p$ 下,所有属于 $S_t^c$ 的类型的收益之和(相当于一个上界):$r_t(p) = \sum_{i \in S_t^c} p(n_{i,p}) \geqslant p(n_{i_t,p})$ \\
        这么做在直观上的理解为给出一个更大的奖励来鼓励对这些不愿意购买的类型的探索,从而获取更多的类型信息。
\end{itemize}

在算法的每一轮中,我们会选择累计奖励加上一个扰动值最大的价格曲线 $p_t = \arg\max_{p \in \bar{\mathcal{P}}} \sum_{\tau=1}^{t-1}r_\tau(p) + \theta_p$,其中扰动 $\theta_p$ 采样于指数分布 $\theta e^{-\theta x}$。

\subsubsection{证明思路}

总遗憾同样可以分解为两部分:离散误差和算法误差。前者是将连续价格曲线空间离散化为 $\bar{\mathcal{P}}$ 带来的误差,后者是由于对抗性设置下的算法在进行决策时带来的遗憾,我们更多的关注后者。

FTPL 算法通常用于完全信息的问题,在该论文设置的对抗性的条件下,我们面对的不是一个完全信息的环境,而论文所构造的“收益”能够在买家不透露类型信息(即不购买)的情况下,能使用一个上界来替代真实的收益(因为 $\sum_{i \in S_t^c} p(n_{i,p}) \geqslant p(n_{i_t,p}) ,\ i_t \in S_t^c$),所以也能为算法提供鼓励其探索这一买家类型的信息。

FTPL 算法的遗憾界限通常会包含一个扰动参数 $\theta$,论文给出了 $O(m^2 \theta T + \theta^{-1}(1+\log |\bar{\mathcal{P}}|))$ 的遗憾界限。我们可以通过优化 $\theta$ 的选择来最小化遗憾值的最终上界:令 $\theta = \sqrt{\frac{1+\log |\bar{\mathcal{P}}|}{m^2 T}}$,则最终的遗憾界限为 $\tilde{O}(m\sqrt{T \log |\bar{\mathcal{P}}|})$。又因为离散后的规模为 $|\bar{\mathcal{P}}| = \tilde{O}\left(\left( \frac{J}{\varepsilon^3} \right)^m \right)$,所以最终的遗憾界限为 $\tilde{O}(m^{3/2}\sqrt{T})$。
